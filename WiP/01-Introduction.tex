%!TEX root = main.tex
\vspace{-10pt}
\section{Introduction\label{sec:intro}}
Precise, instance-level object segmentation is crucial for identifying and tracking objects in a variety of real-world emergent applications of autonomy, including robotics~\cite{Natonek1998}, image organization and retrieval~\cite{Yamaguchi2012}, and medicine~\cite{Irshad2014}. To this end, there has been a lot of work on employing crowdsourcing to generate training data for segmentation, including Pascal-VOC~\cite{Everingham15}, LabelMe~\cite{Torralba2010}, OpenSurfaces~\cite{bell15minc}, and MS-COCO~\cite{Lin2012}. Unfortunately, raw data collected from the crowd is known to be noisy due to varying degrees of worker skills, attention, and motivation~\cite{bell14intrinsic,MDWWelinder2010}. 
\par To deal with these challenges, many have employed heuristics indicative of crowdsourced segmentation quality to pick the best worker-provided segmentation~\cite{Sorokin2008,Vittayakorn2011}. Unfortunately, this approach ends up discarding the majority of the worker segmentations and is limited by what the best worker can do. In this paper, we introduce a novel class of aggregation-based methods that incorporates portions of segmentations from multiple workers into a combined one. To our surprise, despite its intuitive simplicity, we have not seen this class of algorithms described or evaluated in prior work. We evaluate this class of algorithms against existing retrieval-based methods described in Section~\ref{sec:related}. \changes{Our analysis of common worker errors in crowdsourced segmentation shows that workers often segment the wrong semantic objects or erroneously include or exclude large semantically-ambiguous portions of an object in the resulting segmentation. To this end, we propose a clustering-based preprocessing technique that resolves such errors.} %different worker perspectives in multiple segmentations.}
%To resolve semantic ambiguity and mistakes commonly observed in crowdsourced segmentation, we propose a clustering-based preprocessing technique that resolves different worker perspectives in multiple segmentations. 