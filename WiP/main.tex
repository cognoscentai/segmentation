\def\year{2018}\relax
\documentclass[letterpaper]{article} %DO NOT CHANGE THIS
\usepackage{aaai18}  %Required
\usepackage{times}  %Required
\usepackage{helvet}  %Required
\usepackage{courier}  %Required
\usepackage{url}  %Required
\usepackage{graphicx}  %Required
\usepackage{mwe} %minipage
\usepackage{floatrow}
\frenchspacing  %Required
\usepackage[utf8]{inputenc} % allow utf-8 input
\usepackage[T1]{fontenc}    % use 8-bit T1 fonts
\usepackage{hyperref}       % hyperlinks
\usepackage{url}            % simple URL typesetting
\usepackage{booktabs}       % professional-quality tables
\usepackage{amsfonts}       % blackboard math symbols
\usepackage{nicefrac}       % compact symbols for 1/2, etc.
\usepackage{microtype}      % microtypography

\usepackage{amsmath}
\usepackage{subcaption}
\usepackage[font=small,belowskip=-30pt,aboveskip=-20pt]{caption}
\usepackage{mathrsfs} 
\usepackage{cancel}
\usepackage{amsfonts}
\usepackage[titletoc,title]{appendix}
\usepackage{floatrow}
\usepackage[table]{xcolor}
\usepackage[draft]{todonotes}   % notes showed
\usepackage{titlesec}
\titlespacing*{\section}{0pt}{0.5ex}{0.5ex}
% \newcommand{\todo}[2][]{}
% \usepackage{titling}
% \setlength{\droptitle}{-10pt}

\usepackage[]{algorithm2e}
\usepackage{framed}
\newcommand{\papertext}[1]{#1}
\newcommand{\techreport}[1]{}
\newcommand{\agp}[1]{\textcolor{magenta}{Aditya: #1}}
\newcommand{\dor}[1]{\textcolor{blue}{Doris: #1}}
\newcommand{\ads}[1]{\textcolor{green}{Akash: #1}}
% Hide
% \newcommand{\agp}[1]{\textcolor{magenta}{}}
% \newcommand{\dor}[1]{\textcolor{blue}{}}
% \newcommand{\ads}[1]{\textcolor{green}{}}
\setcounter{secnumdepth}{2}

\usepackage{amsopn}
\DeclareMathOperator*{\argmax}{arg\,max}

\newcommand{\sampar}[1]{\vspace{3pt}\noindent{\bf #1}}
\newcommand{\subheading}[1]{\vspace{3pt}\noindent{\bf #1}\\\noindent}

\newcommand{\ta}[1]{\vspace{-3pt}\begin{framed}\vspace{-5pt}\noindent\textit{\underline{Takeaway:} #1}\vspace{-5pt}\end{framed}\vspace{-3pt}}
% usage:   \ta{takeaway text}
\newcommand{\stitle}[1]{\noindent \textbf{#1}}
\newcommand{\takeawaywithqn}[2][]{\vspace{-3pt}\begin{framed}\vspace{-5pt}\noindent\textit{{\bf #1} #2}\vspace{-5pt}\end{framed}\vspace{-3pt}}
% usage:    \ta[question]{takeaway}

\setlength{\pdfpagewidth}{8.5in}  %Required
\setlength{\pdfpageheight}{11in}  %Required
%PDF Info Is Required:
  \pdfinfo{
      /Title (2018 Formatting Instructions for Authors Using LaTeX)
          /Author (AAAI Press Staff)}
% \setlength{\intextsep}{-5pt}  % makes space after figure go away
% \setlength{\belowcaptionskip}{-10pt}
% \setlength{\abovecaptionskip}{-10pt}
\setlength{\textfloatsep}{{10pt plus 0.5pt minus 4.0pt}}
\setlength{\intextsep}{{10pt plus 0.5pt minus 4.0pt}}
\newskip\smallskipamount \smallskipamount=3pt plus 1pt minus 1pt
\newskip\medskipamount   \medskipamount=6pt plus 2pt minus 2pt
\newskip\bigskipamount   \bigskipamount=12pt plus 4pt minus 4pt

\title{Quality Evaluation Methods for Crowdsourced Image Segmentation}
\setlength\titlebox{1.5in}   %squish author title height 
% The \author macro works with any number of authors. There are two
% commands used to separate the names and addresses of multiple
% authors: \And and \AND.
%
% Using \And between authors leaves it to LaTeX to determine where to
% break the lines. Using \AND forces a line break at that point. So,
% if LaTeX puts 3 of 4 authors names on the first line, and the last
% on the second line, try using \AND instead of \And before the third
% author name.
\begin{document}
           \title{Quality Evaluation Methods for Crowdsourced Image Segmentation}
           \author{Doris Jung-Lin Lee, Akash Das Sarma, Aditya Parameswaran}
           \maketitle
           \begin{abstract}
           
           Instance-level image segmentation provides rich information crucial for scene understanding in a variety of real-world applications. In this paper, we propose and evaluate multiple crowdsourced algorithms for the image segmentation problem, including novel worker-aggregation based algorithms and retrieval-based methods based on prior work. We characterize the different types of worker errors observed, and present a clustering algorithm as a preprocessing step that is able to capture semantic errors and filter workers with different semantic perspectives. We demonstrate that aggregation-based algorithms attain better performance than existing retrieval-based approaches, while scaling better with increasing numbers of collected worker segmentations. 
           % Most large-scale image segmentation efforts such as MSCOCO have relied on computing a Jaccard metric against ground truth and retrieving the segmentation provided by the best worker. 
          \end{abstract}
               
          %!TEX root = main.tex
\vspace{-10pt}
\section{Introduction\label{sec:intro}}
Precise, instance-level object segmentation is crucial for identifying and tracking objects in a variety of real-world emergent applications of autonomy, including robotics~\cite{Natonek1998}, image organization and retrieval~\cite{Yamaguchi2012}, and medicine~\cite{Irshad2014}. To this end, there has been a lot of work on employing crowdsourcing to generate training data for computer vision, including Pascal-VOC~\cite{Everingham15}, LabelMe~\cite{Torralba2010}, OpenSurfaces~\cite{bell15minc}, and MS-COCO~\cite{Lin2012}. Unfortunately, raw data collected from crowdsourced image processing tasks is known to be noisy due to varying degrees of worker skills, attention, and motivation~\cite{bell14intrinsic,MDWWelinder2010}. 
\par To deal with these challenges, many have employed heuristics indicative of crowdsourced segmentation quality to pick the best worker-provided segmentation~\cite{Sorokin2008,Vittayakorn2011}. Unfortunately, this approach ends up discarding the majority of the worker responses and is limited by what the best worker can do. In this paper, we introduce a novel class of aggregation-based methods, capable of incorporating portions of responses from multiple workers into a combined segmentation. To our surprise, despite its intuitive simplicity, we have not seen this class of algorithms described or evaluated in prior work. We evaluate performance of this class of algorithms against existing retrieval-based methods. To resolve semantic ambiguity and mistakes commonly observed in crowdsourced segmentation, we propose a clustering-based preprocessing technique that resolves different worker perspectives in multiple segmentations. 
          %!TEX root = main.tex
\section{Related Work\label{sec:related}}
% \todo[inline]{NOTE: Add related works on CV methods for object segmentation, what their problem is and why we need crowdsourcing responses (CITE GraphCut, CRFs)}
% \par As described in the introduction, accurate segmentations are essential for reliable object detection and tracking~\cite{sivic2005discovering,felzenszwalb2008discriminatively,viola2004robust,torralba2004sharing,torralba2003contextual,fe2003bayesian,Bearman2016}, necessitating the use of crowdsourcing to gather training data~\cite{AdrianaKovashka2016}. \agp{the previous sentence can be dropped if you have already talked about use cases for segmentation in the intro -- no need to repeat.}
\begin{figure}[h!]
\centering
\includegraphics[width=\linewidth]{plots/flowchart.png}
\caption{Flowchart summarizing the classes of existing algorithms for image segmentation (blue) and a novel class of algorithms proposed in this paper (yellow). Majority-vote (MV) is colored both blue and yellow, since a common algorithm in crowdsourcing literature, but have not been extensively applied to crowdsourced image segmentation.}
%The crowdsourced approach can be largely classified as retrieval or aggregation-based methods. We further explore hybrid algorithms that makes use of signals that span over multiple categories, described in our technical report.
\label{flowchart}
\end{figure}
% \agp{I would add textbfs to divide the rest of this up into various clearly defined paragraphs,
% one corresponding to each of your highest levels in the hierarchy
% figure you drew. I would also add one line of introduction at the start.}
Many large-scale efforts in image segmentation contain little to no information on the quality characterization and evaluation of the collected dataset~\cite{Torralba2010,MartinFTM01,Li2009,Gurari2015}, which indicates the lack of standardized approaches for quality evaluation in crowdsourced image segmentation. As shown in Figure \ref{flowchart}, we break down the existing quality evaluation methods into several categories:
\stitle{Policy-based methods}
Policy-based quality evaluation methods are specialized segmentation interfaces or workflows that ensures that the data collected are of good quality. Several large-scale crowdsourced segmentation efforts have employed verification techniques within the data collection workflow, such as supervising workers periodically by evaluating worker responses against known ground-truth segmentation during data collection~\cite{Lin2014,Everingham15}. Common scoring functions for characterize the quality of the worker's segmentation against the ground truth includes segmentation accuracy\cite{Everingham15} or Jaccard index \cite{Sameki2015,Gurari2016}. Specialized interfaces for object segmentation have also been developed to ensure high-quality data collection. For example, Song et al. (\citeyear{Song2018}) makes use of worker's responses from four different image segmentation interfaces to derive an aggregated bounding box. Other segmentation workflows also employ vision information to supervise the crowdsourced response\cite{Russakovsky2015,Gurari2016}. 
%Likewise, Gurari et al. (\citeyear{Gurari2016}) employs a voting-based, hybrid approach making use of expert, crowdsourced and biomedical image segmentation algorithms. 
\par Since these policy-based methods are interface-dependent, require expensive expert-drawn ground-truth annotations or vision information, the results are not easily reproducible. In addition, the segmentations collected by the simple click-and-draw interface in many of the large scale segmentation efforts can not be improved with this technique as a post-processing method. Due to the lack of reproducibility, our paper do not compare against these policy-based methods in extensive details.

\stitle{Retreival-based methods}
Retreival-based methods seeks to pick the ``best'' worker segmentation based on some scoring criteria. \cite{Vittayakorn2011} proposed several heuristic scoring functions that included vision information such as edge detection or color mixture to assess the quality of crowdsourced segmentations.
%Other work~\cite{Sameki2015,Sorokin2008} study the problem of image segmentation for biomedical and natural images. 
\cite{Vittayakorn2011} provides a comparison of retrieval-based algorithms for image segmentation. 
\dor{Need to describe Vittyakorn et al in more detail}
Other heuristic approaches that don't require ground truth segmentations include characterizing the user types and studying their click-stream behavior to determine work quality\cite{Cabezas2015,Sameki2015,Sorokin2008}, and comparing the worker responses to features extracted from computer vision algorithms\cite{Vittayakorn2011,Russakovsky2015}. As we demonstrate in this paper, aggregating worker responses not only eliminates the need for costly expert segmentations, but also leads to better results than individual workers alone.

% \par Despite several large-scale efforts to collect crowdsourced image segmentation~\cite{Lin2014,MartinFTM01,Torralba2010,pascal-voc-2012}, most have relied on summarization-based metrics to quantify their segmentation data quality. 
\par \dor{this paragraph is probably not necessary, if so merge a couple sentence into retreival section}The other methods we evaluated draw on prior work in worker quality evaluation adapted to the segmentation context. Starting from the seminal work of Dawid and Skene~\cite{Dawid1979}, applying Expectation-Maximization (EM) to derive both worker accuracies and true answers\agp{not a complete sentence.}. Welinder and Perona~\cite{OCWelinder2010} extend this \agp{what?} to separately model worker quality and the biases applied to binary, multivalued and continuous annotations. Welinder and Perona~\cite{MDWWelinder2010} develop a multidimensional array of worker accuracies and task difficulties by considering object-presence labeling as a noise generation process\agp{do you mean a noisy generative process?}. The objective truth label is captured by a multidimensional quantity of task-specific measurements and  deformed by worker and image related noise, the noisy vector obtained after this process is projected onto the vector of user expertise (which summarizes how well the user perceives each of these measurements), and finally the score is binarized into an inferred label \agp{fairly confusing and really long sentence}.

\stitle{Aggregation-based methods}
Pixel-based majority vote \agp{since you may not have defined what pixel based majority vote is until this point, i would explicitly describe it} was introduced in \cite{Sameki2015} as a way for aggregating expert-bounding boxes to obtain the ground truth, rather than for aggregating worker segmentations. Many have extended this line of work beyond binary classifications by developing EM-like approaches that works on multiple-choice \cite{Karger2013} as well as free-form responses \cite{Lin2012}, but these have not been directly applied to the task of object segmentation.
\par However, while EM algorithms assign probabilities regarding  \textit{how good a worker's segmentation is}, for the task of object segmentation, we are ultimately more interested the end goal of \textit{what is the best segmentation that we can get from these data}. 
To our knowledge, our work is the first to compare various aggregation-based approaches for segmentation quality evaluation extensively against existing approaches, and study how clustering could be used for the issues of multiple perspectives described in \cite{Sorokin2008,Lin2014,Gurari2018}, for crowdsourced image segmentation\agp{what is described? the issue of perspectives? or clustering?}.

\stitle{Vision-based approaches}
\dor{Describe state-of-the-art precise instance-level segmentation (Facebook Detectron, CRFs). Argue that crowd response is still important}

% Since these formal probabilistic models treats worker bounding box as the base quantity for modeling worker quality, the best bounding box that one could derive from such algorithm can only be as good as the best worker bounding box in the dataset.  Even though the annotation probabilities are sufficient for determining the best binary-labels, image information such as overlapping areas would be useful and not account for in these algorithms. We suspect that this is why many area-based metrics are still more commonly used in practice than EM approaches. \dor{we might not need this paragraph}\agp{not sure of the goal of this para either}
          %!TEX root = main.tex
\vspace{-3pt}
\section{Error Analysis\label{sec:error}}
\par On collecting and analyzing a number of crowdsourced segmentations (described in Section~\ref{dataset}), we found that common worker segmentation errors can be classified into three types: (1) \textbf{Semantic Ambiguity:} workers have differing opinions on whether particular regions belong to an object (Figure~\ref{error_examples} left: annotations around `flower and vase' when `vase' is requested); (2) \textbf{Semantic Mistake:} workers annotate the wrong object entirely (Figure~\ref{error_examples} right: annotations around `turtle' and `monitor' when `computer' is requested.); and (3) \textbf{Boundary Imperfection:} workers make unintentional mistakes while drawing the boundaries, either due to low image resolution, small area of the object, or lack of drawing skills (Figure~\ref{tile_demo} left: imprecision around the `dog' object).
\par Quality evaluation methods in prior work have largely focused on minimizing boundary imperfection issues. So, we first describe our novel aggregation-based algorithms designed to reduce boundary imperfections in Section~\ref{precision}. Next, in Section~\ref{perspective}, we discuss a preprocessing method eliminates semantic ambiguities and errors, also observed in prior work~\cite{Sorokin2008,Lin2014,Gurari2018}. We present our experimental evaluation in Section~\ref{sec:experiment}.% and compare them with existing retrieval-based methods in 
%\par %Out of the 46 objects in our dataset, 9 objects suffer from semantic ambiguity, 18 objects from semantic mistakes, and almost all objects suffer from some form of boundary imprecision to varying degrees. 
% \begin{figure*}[ht!]
%     \centering
%     \RawFloats
%     \begin{minipage}[t]{0.65\linewidth}
%     	\vspace{-20pt}
%         \includegraphics[width=\linewidth]{plots/error_examples.png} % second figure itself
%         \caption{Pink is the segmentation from individual workers. Blue solid line delineates the ground truth. The red boxed pointer indicates the task of interest shown to users.}
%         \vspace{-15pt}
%         \label{error_examples}
%     \end{minipage}
%     \begin{minipage}[t]{0.35\linewidth}
%     	\vspace{-25pt}
%         \includegraphics[width=\linewidth]{plots/tile_demo.pdf}
%         \vspace{-35pt}
%         \caption{Segmentation boundaries drawn by five workers in red. Right: Overlaid segmentation creates a masks where the color indicates the number of workers whose segmentation includes the tile region.}
%         \vspace{-20pt}
%         \label{tile_demo}
%     \end{minipage}\hfill
% \end{figure*}
\begin{figure}[h!]
    \centering
    \includegraphics[width=0.85\linewidth]{plots/semantic_error_clust.png}
    \caption{\changes{Top: Examples of semantic ambiguities and mistakes. % where workers display different perspectives on what regions should be included as part of the object. %The pink segmentations are drawn by individual workers. The blue solid line delineates the ground truth. The red boxed pointer is the interface icon indicating the semantic object to be segmented. 
    Bottom: Examples of worker segmentations from different clusters.%, where different colors depict clusters representing different worker perspectives.
    }}
    \label{error_examples}
    \setlength{\abovecaptionskip}{-20pt}
\end{figure}
          \section{Precision-Focused Algorithms\label{sec:precision}}%: Aggregation v.s. Retrieval Comparison}
% \subsection{Retrieval-based methods}
% This class of algorithms tries to identify good and bad workers, and then chooses the best worker segmentation as the output segmentation. In this paper, we look at two different ways of ranking workers and choosing the best worker. First, we use the {\em number of control points}, i.e. number of vertices in a worker's segmentation polygon to rank workers. This is a ranking scheme that~\cite{Vittayakorn2011} showed performs well in practice. Intuitively, workers that have used a larger number of points are likely to have been more precise, and provided a more complex and accurate segmentation. Other heuristic ranking scheme is described in more detail in our technical report~\cite{segmentation-tr}.
\begin{figure}[h!]
\vspace{-10pt}
\centering
\includegraphics[width=0.8\textwidth]{plots/precision_issue_tile_example.png}
\caption{Left: Pink boundaries shows worker segmentations and blue delineates the ground truth. Right: Segmentation boundaries drawn by five workers shown in red. Overlaid segmentation creates a masks where the color indicates the number of workers who voted for the tile region.}
\label{tile_demo}
\end{figure}
% \vspace{-10pt}
At the heart of our aggregation techniques is the ``tile'' data representation, where we logically overlay all workers' segmentations on top of each other, as illustrated in Figure \ref{tile_demo} right, to create non-overlapping discrete tile units. The intuition here is that by splitting the image into tiles, we get finer granularity information than by looking at complete segmentations. This also allows us to aggregate data from multiple workers rather than having to choose a single worker bounding box---enabling the opportunity to choose partial segmentations by fixing one worker's errors via the help from another worker's segmentation. Now, we will describe several algorithms for picking a good set of tiles.

\stitle{Aggregation: Majority Vote Aggregation (MV)} 
\par \noindent Include a tile in the output segmentation if and only if the tile is covered by at least 50\% of all worker segmentations.

\stitle{Aggregation: Expectation-Maximization (EM)}
\par \noindent Unlike MV, which assumes that all workers performs uniformly, EM approaches use worker quality models to infer the likelihood that a tile is part of the ground truth segmentation. The EM algorithm simultaneously estimate both worker qualities and tile likelihoods as hidden variables. Details of the formal derivation and three worker quality models that we have developed can be found in our technical report.

\stitle{Aggregation: Greedy Tile Picking (greedy)} 
\par \noindent Using tile probabilities from EM to estimate intersection area between ground truth and tile, then greedily pick tiles with the largest intersection area ratio until Jaccard score begins to decrease (effectively picking the largest and most probable tiles that should be included first). The Jaccard score is computed between the merged output from the selected set of tiles and MV segmentation.

\stitle{Retrieval: Number of Control Points (num pts)}
\par \noindent Pick the worker segmentation with the largest number of control points around the segmentation boundary (i.e. most precise drawing) as the output segmentation.
          %!TEX root = main.tex
\section{Perspective Resolution\label{sec:perspective}}
% \subsection{Worker Clustering}
To resolve differing worker perspectives, we develop a clustering-based approach.
% is based on the intuition that workers with similar perspectives  will have segmentations that are closer to each other. 
We capture the similarity between a pair of workers by computing the Jaccard score between their segmentations and perform spectral clustering to separate workers into clusters. Figure \ref{error_examples} bottom illustrates how spectral clustering is capable of dividing the worker responses into clusters with meaningful semantic associations, reflecting the crowd's diversity of perspectives in completing same task. Clustering results can be used as a preprocessing step to any of the quality evaluation algorithms by keeping only the segmentations that belong to the largest cluster, which is typically free of any semantic errors.
    % \begin{figure}[h!]
    %   \centering
    %   \includegraphics[width=\textwidth]{plots/20.png}
    %   \caption{Example image showing clustering performed on the same object from Figure \ref{error_examples}.}
    %   \label{cluster_example}
    % \end{figure}
\par In addition, clustering offers an additional benefit of preserving worker's semantic intentions in the case where there are multiple instances of different errors. For example, the mistakened clusters in Figure~\ref{error_examples} bottom right contained semantic concepts ``monitor'' and ``turtle''. While these are considered \textit{bad} annotations for this particular task, this cluster of annotation can provide more data for another semantic segmentation task ``monitor''. A potential future work includes adding additional crowdsourcing tasks for semantic labeling of clusters (which is cheaper and more accurate than segmentation) to enable reuse of annotations across multiple objects and lower the cost of data collection. 
          %!TEX root = main.tex
\section{Experimental Setup\label{sec:experiment_setup}}

\subsection{Dataset Description\label{dataset}}
\par \noindent We collected crowdsourced segmentations from Amazon Mechanical Turk; each HIT consisted of one segmentation task for a specific pre-labeled object in an image. There were a total of 46 objects in 9 images from the MSCOCO dataset~\cite{Lin2014} segmented by 40 different workers each, resulting in a total of 1840 segmentations. Each task contained a keyword for the object and a pointer indicating the object to be segmented. Two of the authors generated the ground truth segmentations by carefully segmenting the objects using the same task and interface. 
\par A sub-sampled dataset was created from the full dataset to determine the efficacy of these algorithms on varying number of worker responses. Every object was randomly sampled worker with replacement. For small worker samples, we average our results over larger number of batches than for large worker samples (which have lower variance, since the sample size is close to the original data size).

\subsection{Evaluation Metrics}
\par \noindent Evaluation metrics used in our experiments measure how well the final segmentation (S) produced by these algorithms compare against ground truth (GT). The most common evaluation metrics used in the literature\cite{Cabezas2015,Sameki2015,Song2018,Lin2014} are area-based methods that take into account the intersection area, $IA=area(S\cap GT)$, or union area, $UA=area(S\cup GT)$ between the worker and ground truth segmentations, including %. Specifically, we use
    $\text{Precision (P)} = \frac{IA(S)}{area(S)}$, 
    $\text{Recall (R)} = \frac{IA(S)}{area(GT)}$, and 
    $\text{Jaccard (J)} = \frac{UA(S)}{IA(S)}$.
    %metrics to evaluate our algorithms.
\subsection{Baseline Algorithms}
\subheading{Retrieval-based Methods}

\par \noindent\textbf{Number of Control Points (num pts)}: This algorithm picks the worker segmentation with the largest number of control points around the segmentation boundary (i.e., the most precise drawing) as the output segmentation \cite{Vittayakorn2011,Sorokin2008}. Intuitively, workers that have used a larger number of points are likely to have been more precise, and provided a more complex and accurate segmentation. 
\par \noindent\textbf{Average worker}: This baseline computes the average Jaccard across all workers, which simulates collecting only a single worker annotation.
\par \noindent\textbf{Best worker}: Selecting the best worker based on Jaccard against ground truth. 

\subheading{Vision-based Methods~\label{sec:vision}}

\par We implement a semi-supervised algorithm that can produce segmentations for arbitrary objects in the absence of large volumes of tailor-made training data. While this algorithm works largely on raw image data, it requires some external help in the form of one ``reference'' segmentation. Intuitively, a rough segmentation can be thought of as a pointer for the algorithm to the relevant regions of the image. The algorithm then uses the color profile of the image to segment out the similarly colored regions of the image that overlap with the reference segmentation. Specifically, we begin by splitting the input image into multiple regions, or {\em tiles} that have the same color using the work of~\cite{felzenszwalb2004efficient}---the desired number of output tiles can be modified using a tuning parameter $k$, to produce finer or coarser tiles.

We used the popular open source segmentation algorithm developed by Felzenszwalb and Huttenlocher~\cite{felzenszwalb2004efficient}. We fixed the smoothing and minimum component size parameters and varied the threshold determining the how refined the segmentation is. As shown in Figure~\ref{vision_example}, larger values for k result in larger components in the result. We overlay the given rough segmentation on top of the color tiles.

\begin{figure}
\vspace{-30pt}
\centering
\includegraphics[width=0.9\linewidth]{plots/vision_tiles.png}
\caption{Example of the vision color tiling for different chosen granularities. Left: Raw image. Vision segmentation with $k=100$(Center) and $k=500$ (Right). Vision tiles with a significant overlap area with the worker segmentation (white boundaries) is selected.}
\label{vision_example}
\end{figure}

\par \noindent\textbf{Average vision}: 
\par \noindent\textbf{Best vision}: 
Now, the algorithm focuses on {\em choosing the right set of tiles based on the given reference segmentation}. 
Intuitively the algorithm picks color tiles that have significant overlap with the given reference segmentation, i.e., returns the union of all tiles for which greater than a certain area threshold of the tile is intersecting with the reference segmentation. We experiment with different granularities for the vision preprocessing as well as scan a variety of tile filtering area thresholds. 

\section{Experimental Results\label{sec:experiment}}
\subsection{Aggregation-based methods perform significantly better than retrieval-based methods (no clustering)}
\begin{figure}[h!]
   \centering
   \includegraphics[trim={0 1pt 4pt 0},clip,width=0.8\linewidth]{plots/Retrieval_vs_Aggregation.pdf}
   \caption{Performance of the original algorithms that do not make use of ground truth information (Left) and ones that do (Right). MV and EM results are so close that they overlay on each other.} %\agp{Explain setup for this. How did you generate this?} \dor{not sure what aditya means?}}%Performance comparison between best-performing retrieval and aggregation-based methods. 
   \label{retrieval_vs_aggregation}   
   % \vspace{-12pt}
   % \setlength{\abovecaptionskip}{-30pt}
   % \setlength{\belowcaptionskip}{-23pt}
\end{figure} 
\npar In Figure~\ref{retrieval_vs_aggregation}, we vary the number of worker segmentations along the x-axis and plot the average Jaccard score on the y-axis across different worker samples of a given size across different algorithms. Figure~\ref{retrieval_vs_aggregation} (left) shows that the performance of aggregation-based algorithms (greedy, EM) exceeds the best-achievable through existing retrieval-based method (Retrieval). Then, in Figure \ref{retrieval_vs_aggregation} (right), we estimate the upper-bound performance of each algorithm by assuming that the `full information' based on ground truth was given to the algorithm. For greedy, the algorithm is aware of all the actual tile overlap and non-overlap areas against ground truth, and does not need to approximate these values. For EM, we consider the performance of the algorithm if the true worker quality parameter values (under our worker quality model) are known. For retrieval, the full information version directly picks the worker with the highest Jaccard similarity with respect to the ground truth segmentation. By making use of ground truth information (Figure~\ref{retrieval_vs_aggregation} right), the best aggregation-based algorithm can achieve a close-to-perfect average Jaccard score of 0.98 as an upper bound, far exceeding the results achievable by any single `best' worker (J=0.91). This result demonstrates that aggregation-based methods are able to achieve better performance by performing inference at the tile granularity, which is guaranteed to be finer grained than any individual worker segmentation. 

\subsection{The performance of aggregation-based methods scale well as more worker segmentations are added.}
\par \noindent Intuitively, larger numbers of worker segmentations result in finer granularity tiles for the aggregation-based methods. The first row in Table~\ref{statsTable} lists the average percentage change in Jaccard between 5-workers and 30-workers samples, demonstrating a monotonically increasing relationship between number of worker segmentations used and the performance. However, retrieval-based methods do not benefit from more segmentations.

\subsection{Clustering as preprocessing improves algorithmic performance.}
\par \noindent The average percentage change between the no clustering and clustering results is shown in Table~\ref{statsTable}. Clustering generally results in an accuracy increase. Since the `full information' variants are already free of semantic ambiguity and errors, clustering does not assist with further improvement. %In particular, we see a greater improvement with clustering preprocessing for algorithms that are not very robust in resolving semantic errors or ambiguity, such as for the \texttt{num pts} retrieval algorithm, than compared to the aggregation-based methods. 
\begin{table}[h!]
   \small
     % \setlength\tabcolsep{1.5pt}
      \begin{tabular}{l|l|l|l|l|l|l|l}
      & \multicolumn{3}{c|}{Retrieval-based} & \multicolumn{4}{l|}{Aggregation-based} \\
      Algorithm         & num pts     & worker    & worker*    & MV     & EM     & greedy   & greedy*   \\ \hline
      Worker Scaling    & -6.30       & -0.25     & 2.58       & 1.63   & 1.64   & 2.16     & 5.59      \\ \hline
      Clustering Effect & 5.92        & 4.00      & -0.02      & 2.05   & 1.38   & 5.55     & -0.06    
      \end{tabular}
      \vspace{10pt}
      \caption{Jaccard percentage change due to worker scaling and clustering. Algorithms with * makes use of ground truth information.}
      \label{statsTable}
\end{table}
\par The clustering preprocessing step can significantly improve performance of algorithms that are not very robust to segmentations with semantic errors or ambiguities, such as the heuristic-based number of points approach. When examining the gap of increase with and without clustering in Figure \ref{cluster_effect}, we find that aggregation-based methods performs better than retrieval-methods exhibits a smaller gap between the performances. This effect is due to aggregation-based method's higher performance in the no cluster case, indicating that it is able to capture some of the semantic ambiguities and errors in the dataset.
\begin{figure}[ht!]
      \centering
      \includegraphics[width=\textwidth]{plots/Effects_of_clustering.pdf}
      \caption{Performance comparisons between averaging over experiments with clustering as a preprocessing step(dotted) and the unclustered cases(solid) for different algorithms.}
      \label{cluster_effect}
\end{figure}


\subsection{Overall: with clustering + our algo > baseline}


\subsection{How well does the inferred worker qualities predict individual worker performance?}
    \subsubsection{Correlation of worker qualities against performance}
     To further investigate how the EM models are performing, we looked at whether the model-inferred worker qualities is indicative of the actual quality of a segmentation. We performed linear fitting independently for each sample-objects and computed the $R^2$ statistics to determine whether worker qualities can accurately predict precision, recall, and Jaccard scores. Visual inspection of the basic worker quality model fitting showed that for objects that suffered from type two errors (semantic ambiguity), the single-parameter worker quality was unable to capture the overbounding behavior, which lead to a low precision and Jaccard. The results are listed in Table \ref{correlation} to highlight how our advanced worker qualities were able to better capture these scenarios. The clustering preprocessing was not performed for the values in Table \ref{correlation} to demonstrate the sole effect of the EM algorithm. Nevertheless, our clustered results also show a similar trend, with an average of $R^2$=0.88 and 0.89 for the GT and GTLSA models across all objects respectively. We also find that in general the linear fit improves as the number of data points increases, which indicates consistency in the fitted model.
    \begin{table}[ht!]
    \small
      \begin{tabular}{ccccccc}
        \hline
           N &   basic &   GT &   GTLSA &   isobasic &   isoGT &   isoGTLSA \\
        \hline
              5 &      0.601 &   0.907 &      0.901 &       0.576 &    0.907 &       0.904 \\
            10 &      0.632 &   0.895 &      0.899 &       0.633 &    0.895 &       0.898 \\
            15 &      0.622 &   0.897 &      0.898 &       0.622 &    0.897 &       0.897 \\
            20 &      0.636 &   0.894 &      0.899 &       0.637 &    0.894 &       0.898 \\
            25 &      0.66  &   0.901 &      0.905 &       0.661 &    0.901 &       0.904 \\
            30 &      0.673 &   0.907 &      \cellcolor{blue!25}0.914 &       0.676 &    0.907 &       \cellcolor{blue!25}0.913 \\
        \hline
      \end{tabular}
        \caption{Linear correlation of worker qualities against ground truth performance for different quality models across different number of workers (N). The lower worker samples exhibit lower $R^2$ due to the variance from smaller number of datapoints for each independent fit. }
        \label{correlation}
    \end{table}
    \vspace{-10pt}
    % \subsubsection{EM performance with different worker quality models}
    %   - why is iso cases not performing as well
    \subsubsection{Best worker quality retrieval}
    One application of worker qualities is that it could be used as an annotation scoring function for retrieving the best quality worker segmentation. We explore this approach by training a linear regression model for every sample-object and use the worker qualities to predict the precision, recall, and Jaccard of individual worker annotations against ground truth. Then, we query the model with the inferred worker quality and retrieve the worker with the best predicted Jaccard. 
    \par The reason why a linear regression model was chosen rather than simply sorting the worker qualities and picking the best is that sorting based on multiple worker qualities (precision, recall, Jaccard) effectively applies equal weighting to all quality attributes, whereas our advanced models are specifically designed to capture cases of false-positives and false-negatives that can yield drastically different recall and precision values. We have tested that the linear regression model performs better on this task that simple sorting is capable of learning the weights that helps it make better predictions. As shown in Table~\ref{bigtable}, the performance of worker-quality based retrieval is comparable the performance other aggregation-based methods. We find that amongst the different worker quality models, advanced worker quality models perform the best, agreeing with our intuition regarding correlation results observed in Table~\ref{correlation}.
    \begin{table}[ht!]
    \small
    \setlength\tabcolsep{3pt}
    \begin{tabular}{lrrrrrr}
      \hline
       algo/N                  &     5 &    10 &    15 &    20 &    25 &    30 \\
      \hline
       num points           & 0.838 & 0.809 & 0.826 & 0.805 & 0.814 & 0.785 \\
       best worker          & 0.891 & 0.902 & 0.905 & 0.909 & 0.912 & 0.914 \\
       \hline
       MV                   & 0.885 & 0.893 & 0.894 & 0.897 & 0.898 & 0.899 \\
       EM[basic]           & 0.884 & 0.893 & 0.894 & 0.897 & 0.898 & 0.899 \\
       EM[GT]              & 0.885 & 0.893 & 0.894 & 0.897 & 0.898 & 0.899 \\
       EM[GTLSA]           & 0.871 & 0.892 & 0.891 & 0.896 & 0.897 & \cellcolor{blue!25} 0.899 \\
       greedy               & 0.888 & 0.896 & 0.896 & 0.902 & 0.905 & 0.906 \\
       wqr[basic]          & 0.878 & 0.877 & 0.877 & 0.877 & 0.878 & 0.878 \\
       wqr[GT]             & 0.884 & 0.885 & 0.885 & 0.885 & 0.887 & 0.887 \\
       wqr[GTLSA]          & 0.874 & 0.881 & 0.883 & 0.885 & 0.886 & \cellcolor{blue!25} 0.887 \\
      \hline
    \end{tabular}
    \caption{Summary of average performance across workers with clustering applied as preprocessing in all algorithms across different number of workers (N). wqr is the abbreviation for best worker quality retrieval methods.}
    \label{bigtable}
    \end{table}
          %!TEX root = main.tex 
\vspace{-10pt}
\section{Conclusion and Future Work}
We identified three different types of errors for crowdsourced image segmentation, developed a clustering-based method to capture the semantic diversity caused by differing worker perspectives, and introduced novel aggregation-based methods that produce segmentations that are more accurate than existing retrieval-based methods.
\par Our preliminary studies show that our worker quality models are good indicators of the actual accuracy of worker segmentations. We also observe that the greedy algorithm is capable of achieving close-to-perfect segmentation accuracy with ground truth information. 
Given the success of aggregation-based methods, including the simple majority vote algorithm, we plan to use our worker quality insights to improve our EM and greedy algorithms. 
We are also working on using computer vision signals to further improve our algorithms.
% Bridging the gap between our current approach to the maximum potential of aggregation-based methods can result in more accurate and perspective-aware crowdsourced segmentation outputs in the future.
\bibliographystyle{named}
\bibliography{reference}
\end{document}
