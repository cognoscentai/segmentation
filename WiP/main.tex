\def\year{2018}\relax
%File: formatting-instruction.tex
\documentclass[letterpaper]{article} %DO NOT CHANGE THIS
\usepackage{aaai18}  %Required
\usepackage{times}  %Required
\usepackage{helvet}  %Required
\usepackage{courier}  %Required
\usepackage{url}  %Required
\usepackage{graphicx}  %Required
\frenchspacing  %Required
\usepackage[utf8]{inputenc} % allow utf-8 input
\usepackage[T1]{fontenc}    % use 8-bit T1 fonts
\usepackage{hyperref}       % hyperlinks
\usepackage{url}            % simple URL typesetting
\usepackage{booktabs}       % professional-quality tables
\usepackage{amsfonts}       % blackboard math symbols
\usepackage{nicefrac}       % compact symbols for 1/2, etc.
\usepackage{microtype}      % microtypography

\usepackage{amsmath}
\usepackage{subcaption}
\usepackage{mathrsfs} 
\usepackage{cancel}
\usepackage{amsfonts}
\usepackage[titletoc,title]{appendix}
\usepackage{floatrow}
\usepackage[table]{xcolor}
\usepackage[draft]{todonotes}   % notes showed
% \newcommand{\todo}[2][]{}

\usepackage[]{algorithm2e}
\usepackage{framed}
\newcommand{\papertext}[1]{#1}
\newcommand{\techreport}[1]{}
\newcommand{\agp}[1]{\textcolor{magenta}{Aditya: #1}}
\newcommand{\dor}[1]{\textcolor{blue}{Doris: #1}}
\newcommand{\ads}[1]{\textcolor{green}{Akash: #1}}
% Hide
% \newcommand{\agp}[1]{\textcolor{magenta}{}}
% \newcommand{\dor}[1]{\textcolor{blue}{}}
% \newcommand{\ads}[1]{\textcolor{green}{}}
\setcounter{secnumdepth}{2}

\usepackage{amsopn}
\DeclareMathOperator*{\argmax}{arg\,max}

\newcommand{\sampar}[1]{\vspace{3pt}\noindent{\bf #1}}
\newcommand{\subheading}[1]{\vspace{3pt}\noindent{\bf #1}\\\noindent}

\newcommand{\ta}[1]{\vspace{-3pt}\begin{framed}\vspace{-5pt}\noindent\textit{\underline{Takeaway:} #1}\vspace{-5pt}\end{framed}\vspace{-3pt}}
% usage:   \ta{takeaway text}
\newcommand{\stitle}[1]{\par \noindent \textbf{#1}}
\newcommand{\takeawaywithqn}[2][]{\vspace{-3pt}\begin{framed}\vspace{-5pt}\noindent\textit{{\bf #1} #2}\vspace{-5pt}\end{framed}\vspace{-3pt}}
% usage:    \ta[question]{takeaway}

\setlength{\pdfpagewidth}{8.5in}  %Required
\setlength{\pdfpageheight}{11in}  %Required
%PDF Info Is Required:
  \pdfinfo{
      /Title (2018 Formatting Instructions for Authors Using LaTeX)
          /Author (AAAI Press Staff)}


\title{Quality Evaluation Methods for Crowdsourced Image Segmentation}

           % The \author macro works with any number of authors. There are two
           % commands used to separate the names and addresses of multiple
           % authors: \And and \AND.
           %
           % Using \And between authors leaves it to LaTeX to determine where to
           % break the lines. Using \AND forces a line break at that point. So,
           % if LaTeX puts 3 of 4 authors names on the first line, and the last
           % on the second line, try using \AND instead of \And before the third
           % author name.

           \author{Authors Removed for Anonymity}

           \begin{document}


           \title{Quality Evaluation Methods for Crowdsourced Image Segmentation}
           \author{Authors removed for anonymity}
           \maketitle
           \begin{abstract}
           Instance-level image segmentation provides rich information crucial for scene understanding in a variety of real-world applications, such as robotics and surveillance. In this paper we propose and evaluate several crowdsourced algorithms, including novel worker-aggregation based algorithms and retrieval-based methods based on prior work, for the image segmentation problem. We also characterize the different types of worker errors observed, and present a clustering algorithm that is able to capture semantic errors and filter workers with different semantic perspectives. We demonstrate that aggregation-based algorithms attains better performance than existing retrieval-based approaches, while scaling better with increasing numbers of collected worker segmentations. 
           % Most large-scale image segmentation efforts such as MSCOCO have relied on computing a Jaccard metric against ground truth and retrieving the segmentation provided by the best worker. 
          \end{abstract}
               
          %!TEX root = main.tex
\vspace{-10pt}
\section{Introduction\label{sec:intro}}
Precise, instance-level object segmentation is crucial for identifying and tracking objects in a variety of real-world emergent applications of autonomy, including robotics~\cite{Natonek1998}, image organization and retrieval~\cite{Yamaguchi2012}, and medicine~\cite{Irshad2014}. To this end, there has been a lot of work on employing crowdsourcing to generate training data for computer vision, including Pascal-VOC~\cite{Everingham15}, LabelMe~\cite{Torralba2010}, OpenSurfaces~\cite{bell15minc}, and MS-COCO~\cite{Lin2012}. Unfortunately, raw data collected from crowdsourced image processing tasks is known to be noisy due to varying degrees of worker skills, attention, and motivation~\cite{bell14intrinsic,MDWWelinder2010}. 
\par To deal with these challenges, many have employed heuristics indicative of crowdsourced segmentation quality to pick the best worker-provided segmentation~\cite{Sorokin2008,Vittayakorn2011}. Unfortunately, this approach ends up discarding the majority of the worker responses and is limited by what the best worker can do. In this paper, we introduce a novel class of aggregation-based methods, capable of incorporating portions of responses from multiple workers into a combined segmentation. To our surprise, despite its intuitive simplicity, we have not seen this class of algorithms described or evaluated in prior work. We evaluate performance of this class of algorithms against existing retrieval-based methods. To resolve semantic ambiguity and mistakes commonly observed in crowdsourced segmentation, we propose a clustering-based preprocessing technique that resolves different worker perspectives in multiple segmentations. 
          %!TEX root = main.tex
\section{Related Work\label{sec:related}}
% \todo[inline]{NOTE: Add related works on CV methods for object segmentation, what their problem is and why we need crowdsourcing responses (CITE GraphCut, CRFs)}
% \par As described in the introduction, accurate segmentations are essential for reliable object detection and tracking~\cite{sivic2005discovering,felzenszwalb2008discriminatively,viola2004robust,torralba2004sharing,torralba2003contextual,fe2003bayesian,Bearman2016}, necessitating the use of crowdsourcing to gather training data~\cite{AdrianaKovashka2016}. \agp{the previous sentence can be dropped if you have already talked about use cases for segmentation in the intro -- no need to repeat.}
\begin{figure}[h!]
\centering
\includegraphics[width=\linewidth]{plots/flowchart.png}
\caption{Flowchart summarizing the classes of existing algorithms for image segmentation (blue) and a novel class of algorithms proposed in this paper (yellow). Majority-vote (MV) is colored both blue and yellow, since a common algorithm in crowdsourcing literature, but have not been extensively applied to crowdsourced image segmentation.}
%The crowdsourced approach can be largely classified as retrieval or aggregation-based methods. We further explore hybrid algorithms that makes use of signals that span over multiple categories, described in our technical report.
\label{flowchart}
\end{figure}
% \agp{I would add textbfs to divide the rest of this up into various clearly defined paragraphs,
% one corresponding to each of your highest levels in the hierarchy
% figure you drew. I would also add one line of introduction at the start.}
Many large-scale efforts in image segmentation contain little to no information on the quality characterization and evaluation of the collected dataset~\cite{Torralba2010,MartinFTM01,Li2009,Gurari2015}, which indicates the lack of standardized approaches for quality evaluation in crowdsourced image segmentation. As shown in Figure \ref{flowchart}, we break down the existing quality evaluation methods into several categories:
\stitle{Policy-based methods}
Policy-based quality evaluation methods are specialized segmentation interfaces or workflows that ensures that the data collected are of good quality. Several large-scale crowdsourced segmentation efforts have employed verification techniques within the data collection workflow, such as supervising workers periodically by evaluating worker responses against known ground-truth segmentation during data collection~\cite{Lin2014,Everingham15}. Common scoring functions for characterize the quality of the worker's segmentation against the ground truth includes segmentation accuracy\cite{Everingham15} or Jaccard index \cite{Sameki2015,Gurari2016}. Specialized interfaces for object segmentation have also been developed to ensure high-quality data collection. For example, Song et al. (\citeyear{Song2018}) makes use of worker's responses from four different image segmentation interfaces to derive an aggregated bounding box. Other segmentation workflows also employ vision information to supervise the crowdsourced response\cite{Russakovsky2015,Gurari2016}. 
%Likewise, Gurari et al. (\citeyear{Gurari2016}) employs a voting-based, hybrid approach making use of expert, crowdsourced and biomedical image segmentation algorithms. 
\par Since these policy-based methods are interface-dependent, require expensive expert-drawn ground-truth annotations or vision information, the results are not easily reproducible. In addition, the segmentations collected by the simple click-and-draw interface in many of the large scale segmentation efforts can not be improved with this technique as a post-processing method. Due to the lack of reproducibility, our paper do not compare against these policy-based methods in extensive details.

\stitle{Retreival-based methods}
Retreival-based methods seeks to pick the ``best'' worker segmentation based on some scoring criteria. \cite{Vittayakorn2011} proposed several heuristic scoring functions that included vision information such as edge detection or color mixture to assess the quality of crowdsourced segmentations.
%Other work~\cite{Sameki2015,Sorokin2008} study the problem of image segmentation for biomedical and natural images. 
\cite{Vittayakorn2011} provides a comparison of retrieval-based algorithms for image segmentation. 
\dor{Need to describe Vittyakorn et al in more detail}
Other heuristic approaches that don't require ground truth segmentations include characterizing the user types and studying their click-stream behavior to determine work quality\cite{Cabezas2015,Sameki2015,Sorokin2008}, and comparing the worker responses to features extracted from computer vision algorithms\cite{Vittayakorn2011,Russakovsky2015}. As we demonstrate in this paper, aggregating worker responses not only eliminates the need for costly expert segmentations, but also leads to better results than individual workers alone.

% \par Despite several large-scale efforts to collect crowdsourced image segmentation~\cite{Lin2014,MartinFTM01,Torralba2010,pascal-voc-2012}, most have relied on summarization-based metrics to quantify their segmentation data quality. 
\par \dor{this paragraph is probably not necessary, if so merge a couple sentence into retreival section}The other methods we evaluated draw on prior work in worker quality evaluation adapted to the segmentation context. Starting from the seminal work of Dawid and Skene~\cite{Dawid1979}, applying Expectation-Maximization (EM) to derive both worker accuracies and true answers\agp{not a complete sentence.}. Welinder and Perona~\cite{OCWelinder2010} extend this \agp{what?} to separately model worker quality and the biases applied to binary, multivalued and continuous annotations. Welinder and Perona~\cite{MDWWelinder2010} develop a multidimensional array of worker accuracies and task difficulties by considering object-presence labeling as a noise generation process\agp{do you mean a noisy generative process?}. The objective truth label is captured by a multidimensional quantity of task-specific measurements and  deformed by worker and image related noise, the noisy vector obtained after this process is projected onto the vector of user expertise (which summarizes how well the user perceives each of these measurements), and finally the score is binarized into an inferred label \agp{fairly confusing and really long sentence}.

\stitle{Aggregation-based methods}
Pixel-based majority vote \agp{since you may not have defined what pixel based majority vote is until this point, i would explicitly describe it} was introduced in \cite{Sameki2015} as a way for aggregating expert-bounding boxes to obtain the ground truth, rather than for aggregating worker segmentations. Many have extended this line of work beyond binary classifications by developing EM-like approaches that works on multiple-choice \cite{Karger2013} as well as free-form responses \cite{Lin2012}, but these have not been directly applied to the task of object segmentation.
\par However, while EM algorithms assign probabilities regarding  \textit{how good a worker's segmentation is}, for the task of object segmentation, we are ultimately more interested the end goal of \textit{what is the best segmentation that we can get from these data}. 
To our knowledge, our work is the first to compare various aggregation-based approaches for segmentation quality evaluation extensively against existing approaches, and study how clustering could be used for the issues of multiple perspectives described in \cite{Sorokin2008,Lin2014,Gurari2018}, for crowdsourced image segmentation\agp{what is described? the issue of perspectives? or clustering?}.

\stitle{Vision-based approaches}
\dor{Describe state-of-the-art precise instance-level segmentation (Facebook Detectron, CRFs). Argue that crowd response is still important}

% Since these formal probabilistic models treats worker bounding box as the base quantity for modeling worker quality, the best bounding box that one could derive from such algorithm can only be as good as the best worker bounding box in the dataset.  Even though the annotation probabilities are sufficient for determining the best binary-labels, image information such as overlapping areas would be useful and not account for in these algorithms. We suspect that this is why many area-based metrics are still more commonly used in practice than EM approaches. \dor{we might not need this paragraph}\agp{not sure of the goal of this para either}
          \section{Preliminaries}
\subsection{Data \& Goals}
We collected crowdsourced segmentation data from Amazon Mechanical Turk where each HIT consisted of one segmentation task for a specific pre-labeled object in the image. There were a total of 46 objects in 9 images from the MSCOCO dataset~\cite{Lin2014}. For each object, we collected segmentation masks from a total of 40 workers. As shown in Figure \ref{interface}, each task contains a semantic keyword and a pointer indicating the object to be segmented. These tasks represent a diverse set of task difficulty (different levels of clutteredness, occlusion, lighting) and levels of task ambiguity. %Given a raw 
\subsection{Evaluation Metrics}
\par Evaluation metrics used in our experiment measures how well the final segmentation (S) produced by these algorithms compare against ground truth (GT). The most common evaluation metric used in literature are area-based methods which take into account the intersection, $IA=area(S\cup GT)$, or union, $UA=area(S\cap GT)$, between the user and the ground truth segmentations. Specifically, we use
    $\text{Precision (P)} = \frac{IA(S)}{area(S)}$, 
    $\text{Recall (R)} = \frac{IA(S)}{area(GT)}$, and 
    $\text{Jaccard (J)} = \frac{UA(S)}{IA(S)}$
    metrics to evaluate our algorithms.
 \techreport{\par A sub-sampled dataset was created from the full dataset to determine the efficacy of these algorithms on varying number of worker responses. Every object was randomly sampled worker with replacement. For small worker samples, we average our results over larger number of batches than for large worker samples (which have lower variance, since the sample size is close to the original data size).}
          %!TEX root = main.tex
\section{Perspective Resolution in Crowdsourced Image Segmentation}
\subsection{Worker Clustering}
Our clustering-based approach is based on the intuition that workers with similar perspectives  will have segmentations that are closer to each other. We capture the similarity between a pair of workers by computing the Jaccard score between their segmentations and perform spectral clustering to separate workers into clusters. Figure \ref{cluster_example} illustrates how spectral clustering is capable of dividing the worker responses into clusters with meaningful semantic associations, reflecting the crowd's diversity of perspectives in completing same task.
    \begin{figure}[h!]
      \centering
      \includegraphics[width=\textwidth]{plots/20.png}
      \caption{Example image showing clustering performed on the same object from Figure \ref{error_examples}.}
      \label{cluster_example}
    \end{figure}
\par Clustering results can also be used as a preprocessing step to any of the quality evaluation algorithms by keeping only the segmentations that belong to the largest cluster, which is typically free of any semantic errors. As shown in Table~\ref{statsTable}, on average, clustering generally results in an increase the resulting algorithmic performance. Since the ground-truth supervised variants are already free of semantic ambiguity and errors, there is minimal improvement resulting from clustering. %In particular, we see a greater improvement with clustering preprocessing for algorithms that are not very robust in resolving semantic errors or ambiguity, such as for the \texttt{num pts} retrieval algorithm, than compared to the aggregation-based methods. 
\par Clustering offers an additional benefit of preserving worker's semantic intentions in the case where there are multiple instances of different errors. For example, in Figure \ref{cluster_example}, the mistakened clusters included semantic concepts ``monitor'' and ``turtle''. While these are considered \textit{bad} annotations for this particular task, this cluster of annotation can provide more data for another semantic segmentation task ``monitor''. A potential direction of future work includes adding an additional crowdsourcing task for semantic labelling of clusters (which is cheaper and more accurate than segmentation) to enable reuse of annotations across objects and lower the cost of data collection. 
          \section{Precision-savvy algorithms: Aggregation v.s. Retrieval Comparison}

%The key insight of aggregation-based methods is that they perform inference at a more fine-grained ``tile'' level, rather than at the bounding box level as is in the case of retrieval-based methods. In this section, we discuss three different algorithms for choosing a ``good'' set of worker tiles in aggregation-based methods.
%In this section, we discuss aggregation based methods which use combine the segmentations from multiple workers and output a single merged segmentation. As described earlier in Section~\ref{sec:crowd}, our aggregation algorithms first transform the set of worker segmentations into tiles, as shown in Fig.~\ref{tile_demo}, and then choose a set of tiles to include in the final segmentation. \todo[inline]{modify figure, or use from prev section if added.} Since aggregation based approaches work at the level of tiles, they are more fine-grained than retrieval-based methods, which work at the granularity of complete worker segmentations.
\subsection{Retrieval-based methods}
This class of algorithms tries to identify good and bad workers, and then chooses the best worker segmentation as the output segmentation. In this paper, we look at two different ways of ranking workers and choosing the best worker. First, we use the {\em number of control points}, i.e. number of vertices in a worker's segmentation polygon to rank workers. This is a ranking scheme that~\cite{Vittayakorn2011} showed performs well in practice. Intuitively, workers that have used a larger number of points are likely to have been more precise, and provided a more complex and accurate segmentation. Other heuristic ranking scheme is described in more detail in our technical report~\cite{segmentation-tr}.
\begin{figure}
\centering
\includegraphics[width=\textwidth]{plots/tile_demo.pdf}
\caption{Left: Red boundaries shows the segmentation boundaries drawn by five workers overlaid on the image. Right: Segmentation boundaries still shown in red. The overlaid segmentation creates a masks where the color indicates the number of workers who voted for the tile region.}
\label{tile_demo}
\end{figure}
\subsection{Aggregation-based methods}
Rather than simply identifying and picking a single worker's segmentation, aggregation-based methods seek to combine multiple workers' segmentations into a single merged segmentation. At the heart of all our aggregation techniques is the following data representation: we logically overlay all workers' segmentations on top of each other within the framework of the overall image. As illustrated in \ref{tile_demo}, the overlaid worker segmentations can be thought of as a Venn diagram that represents a partitioning of the entire image into multiple worker {\em tiles} formed by the intersections of different worker segmentations. We then choose and merge a subset of the tiles to give the final output segmentation\dor{vague}. The intuition here is that by splitting the image into tiles, we get finer granularity information than by looking at complete segmentations. This also allows us to aggregate data from multiple workers rather than having to choose a single worker bounding box---this allows for the potential of choosing the best partial segmentations for an object and joining them, or fixing one worker's errors by taking the help of another worker's segmentation. The problem of choosing a good set of tiles is, however, non-trivial.
Since aggregation based methods are the least studied methods by previous work, we discuss them in further detail in Section~\ref{sec:agg-detailed}.


\subsection{Majority Vote Aggregation (MV)}
The aggregation-based majority vote algorithm examines --- tile, and includes the tile in the output segmentation if and only if the tile is covered by at least 50\% of all worker segmentations.
% We investigated three types of variants for majority vote strategies: picking the top-k, top-percentile tiles of vote counts and picking the tiles that were voted by at least 50\% of the workers. We found that among these majority vote variants, the latter strategy gave the highest accuracy.

\subsection{Expectation-Maximization}
While Majority Vote is a very useful algorithm in practice, it does not distinguish between workers in any way. In reality, however, not all workers are equal. Now, we try to model worker quality, and use worker quality information to infer the likelihood that a tile is part of the ground truth segmentation. Since both, the worker qualities, as well as the likelihoods of tiles being part of the ground truth are hidden quantities, we employ an Expectation-Maximization based approach to simultaneously esimtate both of these sets of quantities. We intuitively describe three worker models that we experiment with below. In our technical report, we formalize the notion of the probability that a set of tiles forms the ground truth, and solve the corresponding maximum likelihood problem, for each of these worker models.

\subheading{Worker quality models.}
We can think of workers as agents that look at each pixel in an image and label it as part of the segmentation, or not. Their actual segmentation is the union of all the pixels that they labeled as being part of their segmentations. Each pixel in the image is also either included in the ground truth segmentation or not included in the ground truth segmentation. We can now model worker segmentation as a set of boolean pixel-level (include or don't include) tasks, each having a ground truth boolean value. Based on this idea, we explore three worker quality models:
\begin{itemize}
\item {\em Basic model:} Each worker is captured by a single parameter Bernoulli model, $<q>$, which represents the probability that a worker will label an arbitrary pixel correctly.
\item {\em Ground truth inclusion model (GT):} Two parameter Bernoulli model $<qp, qn>$, capturing false positive and false negative rates of a worker. This helps to separate between workers that tend to overbound and workers that tend to underbound segmentations.
\item {\em Ground truth inclusion, large small area model (GTLSA):} Four parameter model $<qp_l, qn_l, qp_s, qn_s>$, that distinguishes between false positive and false negative rates for large and small tiles. In addition to capturing overbounding and underbounding tendencies, this model captures the fact that workers tend to make more mistakes on small tiles, and penalizes mistakes on large tiles more heavily.
\end{itemize}


\subsection{Greedy Tile Picking}\dor{the terminology ``overlap'' can be a bit confusing with the abbrev that we chose, since overlap area would be OA (rather than outside area). Maybe introduce it as intersection area or introduce terms ``inside'' and ``outside'' to correspond with the abbrev OA,IA.}
Next, we present a greedy tile picking algorithm that grows the output set of tiles by adding in one tile at a time. Suppose tile $t$, overlaps with the ground truth segmentation with intersection area of $IA(t)$, and has area $OA(t)$ not overlapping with the ground truth. The greedy algorithm sorts tiles in decreasing order of their $\frac{IA(t)}{OA(t)}$ ratio and iteratively adds the next tile to the growing set of output tiles, until the Jaccard value of the current set of tiles will decrease with the next added tile. \dor{explain intuition of why I/O is used.} The key idea behind this algorithm is the following statement\papertext{\todo[inline]{techreport}(proof available in our technical report)}: It can be shown that given a set of tiles, $T$, the tile $t$ that maximizes Jaccard($T\cup t$) score of the union of the set of tiles against the ground truth, is the tile with maximum value of $\frac{IA(t)}{OA(t)}$. The primary challenge with this approach is that we do not know the actual $IA(t)$, $OA(t)$ values for any tile. We implement a heuristic version of this algorithm, where we estimate the intersection area of any tile, $IA(t)$, by using the fraction of workers that have voted for a tile, and greedily maximize for estimated Jaccard value at every step. \papertext{\todo[inline]{techreport}In our technical report, we also discuss variants of this algorithm where we use different techniques to estimate the intersection areas of tiles, resulting in corresponding variants of the greedy algorithm.}

          %!TEX root = main.tex
\section{Results and Discussion\label{sec:results}}

  \subsection{Experimental Setup}

  \subsection{Evaluation Metrics}
   \par  A sub-sampled dataset was created from the full dataset to determine the efficacy of these algorithms on varying number of worker responses, based on Table.\ref{batch_sample}. The number of workers plotted indicates the number of distinct worker segmentation used in the algorithm, note that in the clustered cases, the actual number of segmentation in the cluster may be less than what is cited in the sample. In the subsequent section, we will refer to objects specific to a sample synonymously as objects.
  \begin{table}[ht]
  \centering
  \label{batch_sample}
  \caption{Every object was randomly sampled worker with replacement. For small worker samples, we average our results over larger number of batches than for large worker samples (which have lower variance, since the sample size is close to the original data size).}
  \begin{tabular}{l|llllll}
  \# of workers & 5  & 10 & 15 & 20 & 25 & 30 \\ \hline
  \# of batches & 10 & 8  & 6  & 4  & 2  & 1 
  \end{tabular}
  \end{table}
   \par Evaluation metrics used in our experiment measures how well the final segmentation(S) produced by these algorithms compare against ground truth(GT). The most common evaluation metric used in literature are area-based methods which take into account the intersection, $\mathcal{I}=area(S\cup GT)$, or union, $\mathcal{U}=area(S\cap GT)$, between the user and the ground truth bounding boxes.
    $$\text{Precision (P)} = \frac{\mathcal{I}}{area(S)}$$ \\
    $$\text{Recall (R)} = \frac{\mathcal{I}}{area(GT)}$$ \\
    $$\text{Jaccard (J)} = \frac{\mathcal{U}}{\mathcal{I}}$$ \\
    $$\text{False Positive Rate (FPR)}= \frac{area(S)-\mathcal{I}}{area(GT)}$$\\
    $$\text{False Negative Rate (FNR)}= \frac{area(GT)-\mathcal{I}}{(area(S)-\mathcal{I})+(area(Image)-\mathcal{U})}$$
%%%%%%%%%%%%%%%%%%%%%%%%%%%%%%%%%%%%%%%%%%%%%%%%%%%%%%%%%%%%%%%%%%%%%%%%%%%%%%%%%%%%%%%%%%%%%%%%%%%%%%%%%%%%%%%%%%%%%%%%%%%%%%%%%%%%%%%%%%
  \subsection{What is the difference in performance between retrieval and aggregation-based methods?}
    As shown in Figure \ref{retreival_vs_aggregation}, we performed for the best performing ---- for each family of algo (including clustering)
    \begin{figure}[ht!]
      \centering
      \includegraphics[width=\textwidth]{plots/Retreival_vs_Aggregation.pdf}
      \caption{Performance comparison between best-performing algorithms from retrieval and aggregation-based methods. Color denotes the type of algorithm used and dotted line shows the corresponding algorithm that makes use of ground truth.}
      \label{retreival_vs_aggregation}
    \end{figure}
      \ta{Key advantages of using aggregation-based approaches is due to its performance scalability as more annotations are collected as well as operating at a finer granularity that results in a higher potential upper bound.} Future work on developing better inference models would bring the performance closer to these upper bounds.
      
%%%%%%%%%%%%%%%%%%%%%%%%%%%%%%%%%%%%%%%%%%%%%%%%%%%%%%%%%%%%%%%%%%%%%%%%%%%%%%%%%%%%%%%%%%%%%%%%%%%%%%%%%%%%%%%%%%%%%%%%%%%%%%%%%%%%%%%%%%
  \subsection{How well does the inferred worker qualities predict individual worker performance?}
    \subsubsection{Correlation of worker qualities against performance}
     To further study whether the inferred worker qualities is indicative of the quality of an annotation, we independently performed linear fitting for each sample-objects and computed the $R^2$ statistics to determine whether worker qualities can predict precision, recall, and Jaccard. Table --- cites the ---- of the no clustering cases. Visual inspection of the basic worker quality model fitting showed that for objects that suffered from type two errors, the single-parameter worker quality could not predict the low precision that lead to a low Jaccard. The high correlation score implies that our advanced worker qualities are able to capture the 
    \subsubsection{EM performance with different worker quality models}


    \subsubsection{Best worker quality retrieval}
    One application of worker qualities is that it could be used as an annotation scoring function for retrieving the best quality worker segmentation. We explored this approach by training a linear regression model for every sample-object and use the worker qualities to predict the precision, recall, and Jaccard of individual worker annotations against ground truth. We query the model with the inferred worker quality and retrieve the worker with the best predicted Jaccard. 
    \par We chose to use a linear regression model rather than simply sorting the worker qualities and picking the best is that worker qualities in our advanced models are specifically designed to capture cases of false positives and false negatives that yields very different recall and precision values. Sorting based on multiple worker qualities effectively applies equal weighting to all quality attributes. We have tested that the linear regression model performs better on this task that simple sorting is capable of learning the weights that helps it make better predictions.
    \par 
%%%%%%%%%%%%%%%%%%%%%%%%%%%%%%%%%%%%%%%%%%%%%%%%%%%%%%%%%%%%%%%%%%%%%%%%%%%%%%%%%%%%%%%%%%%%%%%%%%%%%%%%%%%%%%%%%%%%%%%%%%%%%%%%%%%%%%%%%%
  \subsection{How do different families of aggregation based algorithms (EM v.s. Greedy v.s. MV) relate and compare?}
    Given the success of aggregation-based models, we wanted to further study how different algorithms perform compared to one another.

    
    \ta{We show that majority vote, while simple, performs nearly as well as the advanced EM and greedy based approaches.} This is because convergence.



    When using ground truth to estimate intersection areas, we can achieve a Jaccard of 0.983 as an upper bound with 30 workers, which indicates that with better probabilistic estimation of intersection area, aggregation-based methods can achieve close to perfect segmentation outputs, exceeding the results than that of any single `best' worker (J= --- for 30 workers). Suggest precise BB useful for some application.
%%%%%%%%%%%%%%%%%%%%%%%%%%%%%%%%%%%%%%%%%%%%%%%%%%%%%%%%%%%%%%%%%%%%%%%%%%%%%%%%%%%%%%%%%%%%%%%%%%%%%%%%%%%%%%%%%%%%%%%%%%%%%%%%%%%%%%%%%%
  \subsection{How well does clustering resolve multiple perspectives of crowdworkers and improve quality evaluation algorithms?}
    \begin{figure}[ht!]
      \centering
      \includegraphics[width=\textwidth]{plots/20.png}
      \caption{TODO}
      \label{cluster_example}
    \end{figure}
    \begin{figure}[ht!]
      \centering
      \includegraphics[width=\textwidth]{plots/Effects_of_clustering.pdf}
      \caption{TODO}
      \label{cluster_effect}
    \end{figure}
  \par Compared to using a metric-based heuristic to detect and eliminate these errors, clustering has additional benefit of preserving worker's semantic intentions in the case where there are multiple instances of different errors. For example, in the object 20 example above, the mistakened clusters included semantic concepts ``monitor'', ``computer'', ``turtle'', ``computer front''. While these are considered bad annotations for this particular task, this cluster of annotation can provide more data for another semantic task ``monitor''. A potential direction of future work includes adding a additional crowdsourcing task for semantic labelling of clusters (which is cheaper and more accurate than segmentation) to enable reuse of annotations across objects and lowering the cost of data collection. 


%%%%%%%%%%%%%%%%%%%%%%%%%%%%%%%%%%%%%%%%%%%%%%%%%%%%%%%%%%%%%%%%%%%%%%%%%%%%%%%%%%%%%%%%%%%%%%%%%%%%%%%%%%%%%%%%%%%%%%%%%%%%%%%%%%%%%%%%%%
  \subsection{Can vision information be used to improve the crowdsourced responses?}
  We explored the idea of using a vision algorithm to improve aggregation-based quality evaluation methods. 
          %!TEX root = main.tex 
\vspace{-10pt}
\section{Conclusion and Future Work}
We identified three different types of errors for crowdsourced image segmentation, developed a clustering-based method to capture the semantic diversity caused by differing worker perspectives, and introduced novel aggregation-based methods that produce segmentations that are more accurate than existing retrieval-based methods.
\par Our preliminary studies show that our worker quality models are good indicators of the actual accuracy of worker segmentations. We also observe that the greedy algorithm is capable of achieving close-to-perfect segmentation accuracy with ground truth information. 
Given the success of aggregation-based methods, including the simple majority vote algorithm, we plan to use our worker quality insights to improve our EM and greedy algorithms. 
We are also working on using computer vision signals to further improve our algorithms.
% Bridging the gap between our current approach to the maximum potential of aggregation-based methods can result in more accurate and perspective-aware crowdsourced segmentation outputs in the future.
\bibliographystyle{named}
\bibliography{reference}
\end{document}
